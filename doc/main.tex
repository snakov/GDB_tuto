\documentclass{beamer}
\usetheme{Darmstadt}

\title[A GDB Tutorial] {A GDB Tutorial}


\author[S.~Nakov] {Stojche Nakov}

\institute
{
  CS Departement\\
  Princeton University
}
\date[] {Sub-Group meeting, \today}

\begin{document}

\frame{\titlepage}

\frame{\frametitle{GDB: The GNU Project Debugger}
  \begin{block}{GDB overview}
    \begin{itemize}
      {\small
      \item First release in 1986. 
      \item Supports variouse languages: {\bf C, C++, Fortran, Objective-C, Python} (with an extention) etc.
      \item Multiple GUI extensions (DDD, kDgb, Nemiver ...).}
    \end{itemize}
  \end{block}
  \uncover<2->{\begin{block}{GDB's Purpose}
      {\small
        \begin{itemize}
        \item Allows you to see what is going on `inside' another program while it executes.
        \item Makes the porgram stop on specific conditions.
        \item Examine what has happened.
        \item Change things in your program.
        \end{itemize}} 
  \end{block}}
}

\frame{\frametitle{GDB Basics}
  \begin{block} {Compiling}
    {\small
      Must be compiled using the flag ``{\bf -g}''. Also it is recomended
      that the optimization flags are removed and the ``{\bf -Og}'' flag
      is used. \\ To run the program use ``{\bf gdb --args ./exe arg1 arg2}''
      }
  \end{block}
  \vfill
  \uncover<2->{\begin{block}{Basic commands}
      \begin{itemize}
        {\tiny
        \item {\bf run / r} -- Begins running the program. If the program is already active, it restarts it. 
        \item {\bf coninue / c} -- Continues the execution of the program. 
        \item {\bf break / b} -- Sets a breakpint. 
        \item {\bf print / p item} -- Prints an item (value of variable, function, structure, class etc).  
        \item {\bf backtrace / b} -- Shows the calling squence. 
        \item {\bf list / l} -- Prints the code.
        \item {\bf frame / f \# } -- Changes to the given frame.
        }
      \end{itemize}
  \end{block}}
}

\frame{\frametitle{Parallel Debugging}
  \begin{block}{Multi-Threading}
    \small{
    \begin{itemize}
    \item {\bf info threads / t} -- Prints information for each thread.
    \item {\bf thread / t \# } -- Switches to the thread. 
    \item {\bf thread apply all ``cmd'' / t a a ``cmd''} --  Applies ``cmd'' to all threads (usually {\bf backtrace} is used).
    \end{itemize}
    }
    \end{block}
  \vfill
  \uncover<2->{
    \begin{block}{Distributed memory}
      \centering
      \only<2>{{\bf MPI} is {\bf SIMD} parallel model.\\}
      \only<3->{{\bf MPI} is {\bf MIMD} parallel model.\\}
      \vfill
      \small{
      \only<4>{mpirun -np 2 ./exe1 : -np 2 ./exe2 : -np 2 ./exe3} 
      \uncover<5>{mpirun -np 1 {\bf gdb ./exe1} : -np 1 ./exe1 : -np 2 ./exe2 : -np 2 ./exe3}\\}
      \uncover<6>{Alternatively, open one xterm per process: \\ mpirun -np 4 {\bf xterm -e gdb ./exe}}
    \end{block}
  }
}

\frame{\frametitle{GDB Advance}
  \begin{block}{Launching gdb}
`    \begin{itemize}
    \item Execute {\bf gdb}, and then specify the exec file using the command {\bf file}.
    \item Arguments can be supplied to the {\bf run} command.
    \item By setting {\bf ulimit -c unlimited},  and then use {\bf gdb core\_file exec\_file}. 
    \item Attaching to a running process: {\bf gdb attach \$pid}. 
    \item {\bf vgdb}: Valgrind + GDB. Launch valgrind with the
      following arguments: {\bf --vgdb=yes --vgdb-error=0}. \\ Then
      launch gdb with the executable and copy paste the proposed
      commands in the first terminal.
    \end{itemize}
  \end{block}
}

\frame{\frametitle{GDB Advance}
  \begin{block}{}
    \begin{itemize}
    \item Conditional breakpoints: {\bf break if condition | break if not condition}
    \item {\bf watch, rwatch, awatch} commands.
    \item {\bf printf} is availble.
    \item {\bf call} allows to call functions.
    \item {\bf Defining helpers} in the .gdbinit file. 
    \item {\bf info breakpoint} prints the breakpoints. {\bf command \# cmdS end}, will execute all cmdS on each breakpooint.
    \item {\bf set var=value} sets the value of {\bf var}.
    \item {\bf Convenience Functions}, like {\bf \$\_caller\_is}, for example.
    \item {\bf Reverse debugging} with the {\bf record} command. Hint: {\bf set can-use-hw-watchpoints 0}, if planning to use watchpoints in reverse debugging. 
    \end{itemize}
  \end{block}
}

\end{document}
