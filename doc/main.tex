\documentclass{beamer}
\usepackage{listings}

\usetheme{Darmstadt}

\title[A GDB Tutorial] {A GDB Tutorial}


\author[S.~Nakov] {Stojche Nakov}

\institute
{
  CS Departement\\
  Princeton University
}
\date[] {DECADES subgroup meeting, November 11, 2021}

\begin{document}

\frame{\titlepage}

\frame{\frametitle{GDB: The GNU Project Debugger}
  \begin{block}{GDB overview}
    \begin{itemize}
      {\small
      \item First release in 1986. 
      \item Supports various languages: {\bf C, C++, FORTRAN, Objective-C} etc.
      \item Multiple GUI extensions (DDD, kDgb, Nemiver ...).}
    \end{itemize}
  \end{block}
  \uncover<2->{\begin{block}{GDB's Purpose}
      {\small
        \begin{itemize}
        \item Allows you to see what is going on `inside' another program while it executes.
        \item Makes the program stop on specific conditions.
        \item Examine what has happened.
        \item Change things in your program.
        \end{itemize}} 
  \end{block}}
}

\frame{\frametitle{GDB Basics}
  \begin{block} {Compiling}
    {\small
      Must be compiled using the flag ``{\bf -g}''. Also it is recommended
      that the optimization flags are removed and the ``{\bf -Og}'' flag
      is added. \\
      To run the program use ``{\bf gdb --args ./exe arg1 arg2}'' \\
      {\bf Control + x a} for better view. 
      }
  \end{block}
  \vfill
  \uncover<2->{\begin{block}{Basic commands}
      \begin{itemize}
        {\tiny
        \item {\bf run / r} -- Begins running the program. If the program is already active, it restarts it. 
        \item {\bf continue / c} -- Continues the execution of the program. 
        \item {\bf break / b} -- Sets a breakpoint. 
        \item {\bf print / p item} -- Prints an item (value of variable, function, structure, class etc).  
        \item {\bf backtrace / bt} -- Shows the calling sequence. 
        \item {\bf list / l} -- Prints the code.
        \item {\bf frame / f \# } -- Changes to the given frame.
        }
      \end{itemize}
  \end{block}}
}

\frame{\frametitle{Parallel Debugging}
  \begin{block}{Multi-Threading}
    \small{
    \begin{itemize}
    \item {\bf info threads / t} -- Prints information for each thread.
    \item {\bf thread / t \# } -- Switches to the thread. 
    \item {\bf thread apply all ``cmd'' / t a a ``cmd''} --  Applies ``cmd'' to all threads (usually {\bf backtrace} is used).
    \end{itemize}
    }
    \end{block}
  \vfill
  \uncover<2->{
    \begin{block}{Distributed memory}
      \centering
      \only<2>{{\bf MPI} is {\bf SIMD} parallel model.\\}
      \only<3->{{\bf MPI} is {\bf MIMD} parallel model.\\}
      \vfill
      \small{
      \only<4>{mpirun -np 2 ./exe1 : -np 2 ./exe2 : -np 2 ./exe3} 
      \uncover<5>{mpirun -np 1 {\bf gdb ./exe1} : -np 1 ./exe1 : -np 2 ./exe2 : -np 2 ./exe3}\\}
      \uncover<6>{Alternatively, open one xterm per process: \\ mpirun -np 4 {\bf xterm -e gdb ./exe}}
    \end{block}
  }
}

\frame{\frametitle{Launching gdb}
  \begin{block}{}
`    \begin{itemize}
    \item Execute {\bf gdb}, and then specify the exec file using the command {\bf file}.
    \item Arguments can be supplied to the {\bf run} command.
    \item By setting {\bf ulimit -c unlimited},  and then use {\bf gdb exec\_file core\_file}. 
    \item Attaching to a running process: {\bf gdb attach \$pid}. 
    \item {\bf vgdb}: Valgrind + GDB. In the first terminal, launch
      valgrind and add these arguments: {\bf --vgdb=yes
        --vgdb-error=0}. \\ Then open an other terminal, start gdb and
      copy paste the proposed commands in the first terminal.
    \end{itemize}
  \end{block}
}

\begin{frame}[fragile]
  \frametitle{Improve gdb's output}
  \begin{block}{}
    \begin{itemize} 
      \item Hit {\bf Control + x 2}, hit it again! And AGAIN!! 
      \item {\bf printf} is available.
      \item User define helpers in the {\bf .gdbinit} file.  First add
        one of the next two lines to your global gdbinit
        ($\sim$/.initgdb):\\
        \small{
          \begin{lstlisting}
# disable safe checks
1) set auto-load safe-path /

# a bit safer
2) add-auto-load-safe-path  /path/to/your/working/gdbinit 
        \end{lstlisting}}
    \end{itemize}
  \end{block}
\end{frame}

\begin{frame}[fragile]
  \frametitle{A local .gdbinit file example}
  \small{
  \begin{lstlisting}
# break on main automatically
break main

#define a function
define my_print
  set language c # tell gdb what language to use
  printf "Here comes the first arg: <%d> \n" , $arg0
  set $n = 0
  while $n != $arg1
    printf "Second argument is not %d \n", $n
    set $n = $n + 1
  end
  printf "Second argument is  %d !!!!\n", $arg1
end
  \end{lstlisting}
}
\end{frame}

\begin{frame}[fragile]\frametitle{GDB advanced breaks}
  \begin{block}{}
    \begin{itemize}
    \item Conditional breakpoints: {\bf break if condition | break if not condition}
    \item {\bf watch, rwatch, awatch} watch the memory!
    \item {\bf info breakpoint} prints the breakpoints (including wathe).
    \item {\bf command \# instruction 1\; instruction2\; end}, will execute all instruction each time that breakpoint is accessed.
    \end{itemize}
    \begin{lstlisting}
      command 2
      # suppresses the normal output
      silent 
      print var
      next
      print var2
      end
    \end{lstlisting}
  \end{block}
\end{frame}



\frame{\frametitle{Other GDB's advance toys}
  \begin{block}{}
    \small{
    \begin{itemize}
    \item {\bf call} allows to call functions/methods defined in the code.
    \item {\bf set var variable=value} sets the value of {\bf variable}.
    \item {\bf Convenience Functions}, (help function), like {\bf \$\_caller\_is}, for example.
    \item {\bf Reverse debugging} Yes it exists. Here is how to use it:
      \begin{itemize}
      \item Break on main, and then type {\bf record}. All following instructions are recorded.
      \item When you hit a breakpoint or the program crashes, use {\bf reverse-continue, reverse-next } etc. 
      \item  Hint: {\bf set can-use-hw-watchpoints 0}, if planning to use watchpoints in reverse debugging. 
    \end{itemize}
    \end{itemize}}
  \end{block}
}

\frame{
  \centering
  \huge{Thank you for your attention}
}

\end{document}
